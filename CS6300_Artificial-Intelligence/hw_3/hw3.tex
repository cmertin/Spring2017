\documentclass[12pt]{article}

\usepackage{times}
\usepackage{notes}
\usepackage{url}
\usepackage{graphicx}
\usepackage{amsmath}
\usepackage{forest}
\usepackage{xcolor}
\usepackage{float}
\setlength{\textwidth}{6.5in}
\setlength{\textheight}{8.9in}
\setlength{\oddsidemargin}{0.0in}
\setlength{\topmargin}{0.05in}
\setlength{\headheight}{-0.05in}
\setlength{\headsep}{0.0in}

\usetikzlibrary{decorations.markings}

\newsavebox\mybox
\savebox\mybox{%
  \tikz{
    \draw[ultra thick,red] (-4pt,-4pt) -- (4pt,4pt);
    \draw[ultra thick,red] (-4pt,4pt) -- (4pt,-4pt);
  }%
}  

\tikzset{
myedge/.style={
  decoration={
   markings,
   mark=at position 0.5 with \node {\usebox\mybox};
  },
  postaction=decorate
  }
}

\begin{document}

\begin{center}
{\bf CS 6300} \hfill {\large\bf HW3: Expectimax and Probability} \hfill {\bf Due February 7, 2017}
\end{center}

Please use the \LaTeX\ template to produce your writeups. See the
Homework Assignments page on the class website for details.  Hand in
at: \url{https://webhandin.eng.utah.edu/index.php}.

\section{Expectimax}

Consider this modified form of blackjack in Utah's only casino: this
CS 6300 Artificial Intelligence class!  By playing this game, you
can add (or subtract) points to your course grade.

\begin{itemize}

\item There are three card values in an infinite deck: 5, 10, and 11.
  They are equally probable.

\item There are two actions available: hit (draw a card), or stay
  (game ends).

\item The total value of cards must be less than or equal to 21.  If
  the value is more than 21, the player is bust and the game ends.

\item The player must stay if the value is 21.

\item The payoff schedule is:

      \begin{center}
      \begin{tabular}{c|c}\hline
        Hand value & Payoff \\ \hline
        0-14       & 0      \\
        15-19      & 3      \\
        20         & 9      \\
        21         & 12     \\
        Bust       & -6     \\ \hline
      \end{tabular}
      \end{center}

\end{itemize}

\begin{enumerate}

\item Suppose the player's first card is an 11.  Draw the expectimax
  tree using a drawing program.  Show chance, max, and terminal nodes,
  and work out their values.  What is the optimal strategy?

\begin{figure}[h]
\centering
\scalebox{1}
{%
\begin{forest}
[11
    [5 (16) {\color{blue}{+3}}
        [5 (21) {\color{green}{+12}}
        ]
        [10 (26) {\color{red}{-6}}
        ]
        [11 (27) {\color{red}{-6}}
        ]
    ]
    [10 (21) {\color{green}{+12}}
    ]
    [11 (22) {\color{red}{-6}}
    ]
]
\end{forest}
}
\end{figure}
The optimal strategy would be to only draw one card and then stay. It has a 1/3 chance of going bust, but 2/3 chance of some payout. The average ``winnings'' of the 2nd level is zero after drawing 5. Stoping after the first gives an average winning of 11/2.

\item Unfortunately, you are playing with an unscrupulous dealer (one
  of the TAs) who after a fair first card then gives you the worst
  possible card thereafter.  Draw the expectimax tree, and work out
  values for the nodes.  What is the optimal strategy?


\begin{figure}[H]
\centering
\scalebox{1}
{%
\begin{forest}
[0
    [5 (5) {\color{blue}{+0}}
        [5 (10) {\color{blue}{+0}}
            [5 (15) {\color{blue}{+3}}
                [10 (25) {\color{red}{-6}}]
                [11 (26) {\color{red}{-6}}]
            ]
        ]
    ]
    [10 (10) {\color{blue}{+0}}
        [5 (15) {\color{blue}{+3}}
            [10 (25) {\color{red}{-6}}]
            [11 (26) {\color{red}{-6}}]
        ]
    ]
    [11 (11) {\color{blue}{+0}}
        [5 (16) {\color{blue}{+3}}
            [10 (26) {\color{red}{-6}}]
            [11 (27) {\color{red}{-6}}]
        ]
    ]
]
\end{forest}
}
\end{figure}

The optimal play strategy is to top at the second dealt card if given a 10 or an 11, but to go one card after after the second 5. This guarenttees a win of +3.

\item Unforunately, interest in AI blackjack is decreasing, and the
  instructor asks for your help in reworking the game by increasing
  the payoff for a value of 21 (currently $x=12$).  Suppose a player
  has a hand value of 16.  What should the smallest value of $x$ be
  (an integer) so that it is advantageous to hit rather than stay?


\begin{figure}[h]
\centering
\scalebox{1}
{%
\begin{forest}
[16 {\color{blue}{+3}}
    [5 (21) {\color{green}{+12}}
    ]
    [10 (26) {\color{red}{-6}}
    ]
    [11 (27) {\color{red}{-6}}
    ]
]
\end{forest}
}
\end{figure}

The average in the above figure is 0 gain from ``hitting'' with a 16 in hand. Therefore, the smallest that it would have to be is +15. This will have an average of +1 overall in playing.

\end{enumerate}

\clearpage

\section{Probability}

Marijuana legalization has been in the news, and one of the states is
having a gubernatorial election.  The Libertarian candidate (random
variable $L$) is more likely to legalize marijuana (random variable
$M$) than the other candidates, but legalization may happen if any
candidate is elected.  The probabilities are modeled below.

\begin{center}
\begin{tabular}{c@{\hspace*{0.5in}}c}
\begin{tabular}{|c|c|c|} \hline
       & $+l$  & $-l$ \\ \hline
$P(L)$ & 0.1   & 0.9  \\ \hline
\end{tabular} &
\begin{tabular}{|c|c|c|} \hline
     & $P(+m | L)$ & $P(-m | L)$ \\ \hline
$+l$ & 0.667       & 0.333     \\ \hline
$-l$ & 0.25        & 0.75      \\ \hline
\end{tabular} \\
Libertarian governor elected & Marijuana legalized
\end{tabular}
\end{center}

\begin{enumerate}

\item What is $P(+m)$?

\begin{align*}
P(+m) &= P(+m|+l)P(+l) + P(+m|-l)P(-l)\\
      &= 2/3 \cdot 1/10 + 1/4 \cdot 9/10 = 7/24
\end{align*}
\item What is $P(+l | +m)$?

\begin{align*}
P(+l|+m) &= \frac{P(+m|+l)P(+l)}{P(+m|+l)P(+m)}\\
         &= \frac{2/3 \cdot 1/10}{7/24} = 8/35
\end{align*}

\item Fill in the joint distribution table below.

\begin{center}
\begin{tabular}{|c|c||c|} \hline
$L$  & $M$   & $P(L, M)$ \\ \hline \hline
$+l$ & $+m$  & 1/15   \\ \hline
$+l$ & $-m$  & 1/30 \\ \hline
$-l$ & $+m$  & 9/40  \\ \hline
$-l$ & $-m$  & 27/40 \\ \hline
\end{tabular}
\end{center}

\item More information is provided with new random variables $B$
  (balanced budget) and $A$ (workplace absenteeism).

\begin{center}
\begin{tabular}{c@{\hspace*{0.5in}}c}
\begin{tabular}{|c|c|c|} \hline
     & $P(+b | M)$ & $P(-b | M)$ \\ \hline
$+m$ & 0.4         & 0.6         \\ \hline
$-m$ & 0.2         & 0.8         \\ \hline
\end{tabular} &
\begin{tabular}{|c|c|c|} \hline
     & $P(+a | M)$ & $P(-a | M)$ \\ \hline
$+m$ & 0.75        & 0.25        \\ \hline
$-m$ & 0.5         & 0.5         \\ \hline
\end{tabular} \\
Balanced Budget & Absenteeism
\end{tabular}
\end{center}

Fill in the joint distribution table below.

\begin{center}
\begin{tabular}{cc}
\begin{tabular}{|c|c|c|c||c|} \hline
$L$  & $M$   & $B$  & $A$  & $P(L, M, B, A)$ \\ \hline \hline
$+l$ & $+m$  & $+b$ & $+a$ &                 \\ \hline
$+l$ & $+m$  & $+b$ & $-a$ &                 \\ \hline
$+l$ & $+m$  & $-b$ & $+a$ &                 \\ \hline
$+l$ & $+m$  & $-b$ & $-a$ &                 \\ \hline
$+l$ & $-m$  & $+b$ & $+a$ &                 \\ \hline
$+l$ & $-m$  & $+b$ & $-a$ &                 \\ \hline
$+l$ & $-m$  & $-b$ & $+a$ &                 \\ \hline
$+l$ & $-m$  & $-b$ & $-a$ &                 \\ \hline
\end{tabular} &
\begin{tabular}{|c|c|c|c||c|} \hline
$L$  & $M$   & $B$  & $A$  & $P(L, M, B, A)$ \\ \hline \hline
$-l$ & $+m$  & $+b$ & $+a$ &                 \\ \hline
$-l$ & $+m$  & $+b$ & $-a$ &                 \\ \hline
$-l$ & $+m$  & $-b$ & $+a$ &                 \\ \hline
$-l$ & $+m$  & $-b$ & $-a$ &                 \\ \hline
$-l$ & $-m$  & $+b$ & $+a$ &                 \\ \hline
$-l$ & $-m$  & $+b$ & $-a$ &                 \\ \hline
$-l$ & $-m$  & $-b$ & $+a$ &                 \\ \hline
$-l$ & $-m$  & $-b$ & $-a$ &                 \\ \hline
\end{tabular}
\end{tabular}
\end{center}

\item Compute the following.

      \begin{enumerate}

      \item $P(+b | +m) = 4/10$ (directly from conditional probability)

      \item $P(+b | +m, +l) = 4/10$ (from conditional probability since $+l$ is independent)

      \item $P(+b)$

      \item $P(+a | +b)$

      \end{enumerate}

\end{enumerate}

\end{document}
