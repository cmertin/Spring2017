\documentclass[12pt]{article}

\usepackage{times}
\usepackage{graphicx}
\usepackage{amsmath}
\usepackage{url}

\setlength{\textwidth}{6.5in}
\setlength{\textheight}{8.9in}
\setlength{\oddsidemargin}{0.0in}
\setlength{\topmargin}{0.05in}
\setlength{\headheight}{-0.05in}
\setlength{\headsep}{0.0in}

\newcommand{\indep}{\perp\!\!\!\perp}

\begin{document}

\begin{center}
{\bf CS 6300} \hfill {\large\bf HW9: VPI and HMMs \hfill Due April 18, 2017}
\end{center}

\noindent
Please use the \LaTeX\ template to produce your writeups. See the
Homework Assignments page on the class website for details.  Hand in
at: \url{https://webhandin.eng.utah.edu/index.php}.

\section{Decision Networks and VPI}

\begin{center}
\includegraphics[width=6in]{prob1.png}
\end{center}

Your parents are visiting you for graduation.  You are in charge of
picking them up at the airport.  Their arrival time ($A$) might be
early ($e$) or late ($l$).  You decide on a time ($T$) to go to the
airport, also either early ($e$) or late ($l$).  Your sister ($S$) is
a noisy source of information about their arrival time. The
probability values and utilities are shown in the tables above.

Compute $P(S), P(A|S)$ and compute the quantities below. 

\begin{align*}
P(S=e) &= 1.2 \Rightarrow \frac{1.2}{0.8 + 1.2} = 0.6\\
P(S=l) &= 0.8 \Rightarrow \frac{0.8}{0.8 + 1.2} = 0.4
\end{align*}

\begin{align*}
P(A=e|S=e) &= \frac{P(S=e|A=e)P(A=e)}{P(s=e)} = \frac{0.8(0.5)}{0.6} = 2/3\\
P(A=l|S=e) &= 1 - 2/3 = 1/3\\
P(A=l|S=l) &= \frac{P(S=l|A=l)P(A=l)}{P(s=e)} = \frac{0.6(0.5)}{0.4} = 3/4\\
P(A=e|S=l) &= 1 - 3/4 = 1/4
\end{align*}

\begin{enumerate}

\item $EU(T=e) = 0.5(600) + 0.5(300) = 450$

\item $EU(T=l) = 0.5(0) + 0.5(600) = 300$

\item $MEU( \{ \} ) = \max_{T}EU(T,A) = 450$

\item Optimal action with no observations

$ T = e$

\end{enumerate}

\noindent
Now we consider the case where you decide to ask your sister for input.

\begin{enumerate}

\item $EU(T=e | S = e)$

$ = P(A=e|S=e)EU(T=e|A=e) + P(A=l|S=e)EU(T=e|A=l)$

$ =  2/3(600) + 1/3(300) = 500$

\item $EU(T=l | S = e)$

$ = P(A=e|S=e)EU(T=l|A=e) + P(A=l|S=e)EU(T=l|A=l)$

$ = 2/3(0) + 1/3(600) = 200$

\item $MEU( \{ S=e \} )$

$ = \max_{T}\sum_{A}P(A|S=e)EU(T|A) = 500$

\item Optimal action with observation $\{S = e\}$

$T = e$

\item $EU(T = e | S = l)$

$ = P(A=l|S=l)EU(T=e,A=l) + P(A=e|S=l)EU(T=e,A=e)$

$ = 3/4(300) + 1/4(600) = 375$

\item $EU(T = l | S = l)$

$ = P(A=l|S=l)EU(T=l,A=l) + P(A=e|S=l)EU(T=l,A=e)$

$ = 3/4(600) + 1/4(0) = 450$

\item $MEU( \{ S=l \} )$

$\max_{T}\sum_{A} P(A|S=l)EU(T) = 450$

\item Optimal action with observation $S = l$

$T = l$

\item $VPI(S)$

$ = MEU(\{S=e\})P(S=e) + MEU(\{S=l\})P(S=l) - MEU(\{\})$

$ = 500(0.6) + 450(0.4) - 450 = 30$

\end{enumerate}

\section{Wherefore art thou Romeo?}

Romeo and Juliet are two lovesick robots; they function best when each
knows where the other is.  Romeo has become lost, and is trying to
figure out where he is so he can tell Juliet.  Romeo is on the grid
below, which also lists transition probabilities and properties of the
sensors.  At each step, Romeo senses, and then transitions to an
adjacent room to get to the next time step.  Romeo observed the
following evidence while wandering in grief over his inability to tell
Juliet where he is: 2 walls, 2 walls, 3 walls.

\begin{description}

\item[Forward Algorithm] Compute the most likely location, given the evidence.

The forward algorithm is given by

\begin{align*}
  P(x_{t+1}|e_{1:t+1}) &= P(e_{1:t+1}|x_{t+1})\sum_{x_{t}}P(x_{t+1}|x_{t})P(x_{t},e_{1:t})
\end{align*}

We can use this for each time step to calculate the probability of each square. The {\em possible} squares for the robot to be in for $t = 1$ are $\{B\}$ and $\{D,G,F\}$, where $B$ would be where number of walls were wrong and $\{D,G,F\}$ being a correct reading. From here, we can calculate the probability of each square for $t = 1$.

\begin{align*}
  P(B) &= 1/4[1(1/7) + 1(1/7)] = 1/14\\
  P(D) &= 3/4[1/3(1/7) + 1/2(1/7)] = 5/56\\
  P(G) &= 3/4[1/2(1/7) + 1/2(1/7)] = 3/28\\
  P(F) &= 3/4[1(1/7) + 1/2(1/7)] = 3/7
\end{align*}

For $t = 2$ we get $\{B\}$ and $\{D,G,F\}$ again, except this time each room isn't equally likely, as we're constrained to being in the previous states for $t = 1$. This gives

\begin{align*}
  P(B) &= 1/4[1/2(5/56)] = 5/448\\
  P(D) &= 3/4[1/3(1/14) + 1/2(3/28)] = 13/224\\
  P(G) &= 3/4[1/2(5/56) + 1/2(3/7)] = 87/448\\
  P(F) &= 3/4[1/2(3/28)] = 9/224
\end{align*}

Finally, for $t=3$ we have the following sets $\{D,G,F\}$ and $\{A, C, E\}$, where again we're constrained from starting at $\{B,D,G,F\}$ for $t=2$. 

\begin{align*}
  P(D) &= 1/4[1/3(5/448) + 1/2(87/448)] = 271/10752 \approx 0.0252\\
  P(G) &= 1/4[1/2(13/224) + 1/2(9/224)] = 11/896 \approx 0.0123\\
  P(F) &= 1/4[1/2(87/448)] = 87/3584 \approx 0.0243\\
  P(A) &= 3/4[1/3(5/448)] = 5/1792 \approx 0.0028\\
  P(B) &= 3/4[1/3(5/448)] = 5/1792 \approx 0.0028\\
  P(E) &= 3/4[1/2(9/224)] = 27/1792 \approx 0.0151\\
\end{align*}

The largest is $P(D)$ meaning the robot is most likely in square $D$

\item[Viterbi Algorithm] Compute the most likely sequence of steps he
  took in the maze for the above evidence.

The equation for the Viterbi Algorithm is similar to that of the Forward Algorithm except for a minor change. It can be seen below

\begin{align*}
  P(x_{t+1}|e_{1:t+1}) &= P(e_{1:t+1}|x_{t+1})\max_{x_{t}}P(x_{t+1}|x_{t})P(x_{t},e_{1:t})
\end{align*}

For $t=1$ we get the same set of states, $\{B\}$ and $\{D,G,F\}$, giving

\begin{align*}
  P(B) &= 1/4[1(1/7)] = 1/28\\
  P(D) &= 3/4[1/2(1/7)] = 3/56\\
  P(G) &= 3/4[1/2(1/7)] = 3/56\\
  P(F) &= 3/4[1(1/7)] = 3/28
\end{align*}

For $t=2$ we get the same set of states, $\{B\}$ and $\{D,G,F\}$, giving

\begin{align*}
  P(B) &= 1/4[1/2(3/56)] = 3/448\\
  P(D) &= 3/4[1/2(3/56)] = 9/448\\
  P(G) &= 3/4[1/2(3/28)] = 9/224\\
  P(F) &= 3/4[1/2(3/56)] = 9/448
\end{align*}

Finally, for $t=3$ we have the following sets $\{D,G,F\}$ and $\{A, C, E\}$, where again we're constrained from starting at $\{B,D,G,F\}$ for $t=2$.

\begin{align*}
  P(D) &= 1/4[1/2(9/224)] = 9/1792 \approx 0.0050\\
  P(G) &= 1/4[1/2(9/448)] = 9/3584 \approx 0.0025\\
  P(F) &= 1/4[1/2(9/224)] = 9/1792 \approx 0.0050\\
  P(A) &= 3/4[1/3(3/448)] = 3/1792 \approx 0.0017\\
  P(C) &= 3/4[1/3(3/448)] = 3/1792 \approx 0.0017\\
  P(E) &= 3/4[1/2(9/448)] = 37/3584 \approx 0.0075
\end{align*}

With the highest being block $E$, meaning the robot has the highest probability of being in block $E$.


\end{description}

\clearpage

\begin{center}
\includegraphics[height=7in]{prob2.png}
\end{center}

\end{document}


